Goal of this project is to create a flexible application which allows user to view enviroment variables on a website interface and controll the enviromnet from interface within a small chamber with actuating technology based on user\textquotesingle{}s choice.

The base of application should conist of sensors able to sense the following\+:
\begin{DoxyItemize}
\item Light Intensity -\/ This would be used as a hueristic for controlling light actuator
\item Temperature
\item Humidity
\item Co2 or e\+Co2
\item T\+V\+OC
\end{DoxyItemize}

All sensors should utilize one protocol if possible this should be I2\+C/\+S\+M\+B\+US. S\+PI Would be faster, however, extension would be required for R\+Pi which would complicate design.

A base class is going to be created. This class is going to contain the following virtual functions to be overriden by each sensor class when inherting.
\begin{DoxyItemize}
\item Initialize -\/ Opens the file descriptor to the sensor and does initial writes to sets up the sensor for reading operations.
\item Reset -\/ Writes to sensor to set up initial state.
\item Close Device -\/ Calls Reset function and closes the file descriptor
\end{DoxyItemize}

The class is going to also following private variables\+:
\begin{DoxyItemize}
\item m\+\_\+p\+I2\+C\+Driver -\/ a pointer to I2C driver object.
\item m\+\_\+fd -\/ File Descriptor for I2C device.
\end{DoxyItemize}

The I2C driver functions is going to be implemented as signle object, only one instance of the object is going to be initialized and reference to this object is going to be passed in initialization stage.

is going to contain methods for plain I2C writes as well as S\+M\+B\+US

Flexibility is key, relay is going to be used to allow for user to connect desired actuaiting. 4 Channels is the minimmum and it is going to allow for the following\+:
\begin{DoxyItemize}
\item Lights
\item Heating Element
\item Fan
\item Water Pump
\end{DoxyItemize}

The relay is going to be controlled by G\+P\+IO Pins.

The application must be multi-\/threaded. To avoid complexity in terms of concurency and communications\+:
\begin{DoxyEnumerate}
\item All threads must be contained in a single process
\item Call-\/backs is going to be used where possible.
\end{DoxyEnumerate}

Threads is going to idle most of the time and not execute constantly. This is can be achievied by interrupts, timers and blocking read operations.

Delay may be used only if it is absolotuley required(does not infulence/delay any other tasks), otherwise timers are the preffered choice.

Ideally an interrupt service routines is going to write to handle sensor interrupts. If Interrupt architecture cannot be realized then a a sampling is going to be implemented by a timer with an event function for whole I2C Bus. The sampling rate in such case is going to be at least 1\+Hz but preferablly 4\+Hz (Average human reaction time).

As soon as the timer fires the timer event function executes. The Timer event is going to do the following\+:
\begin{DoxyEnumerate}
\item Gather data from each sensor by turn
\item Use controller callback function which sends data to server and executes the actuation loop.
\end{DoxyEnumerate}

\hyperlink{classController}{Controller} object is going to contain the callback function for sampler object to use as well as set of functions to Tx \& Rx data to \& from web-\/server. \hyperlink{classController}{Controller} is going to contain data structures holding latest environment data from sampler and hueristics passed from the user by the server. \hyperlink{classController}{Controller} is going to contain relay object(s) and have set of functions to controll them.

j\+Query -\/$>$ flask server -\/$>$ nginx -\/$>$ C\+GI handler -\/$>$ C\+PP

\hyperlink{classController}{Controller} thread should initialize the sampler and controller object, start sampling and go into a while loop which should execute based on an exit flag which is going to be able to controlled from web server. In this while loop thread is going to read data from the server. The Read operation for the server should be blocking and as soon as the messeage has been recieved it is going to be processed by striping the opcode and this opcode with the corresponding value should be passed to the event handler function.

Project is built on these technologies\+:


\begin{DoxyItemize}
\item Git
\begin{DoxyItemize}
\item Source control
\end{DoxyItemize}
\item C\+Make
\begin{DoxyItemize}
\item Used as build tool
\item Doxygen generation
\end{DoxyItemize}
\item Doxygen
\begin{DoxyItemize}
\item For documentation and A\+PI reference generation
\item Generates wiki from markdown files
\item Graphviz -\/ for creating U\+ML diagrams with doxygen
\end{DoxyItemize}
\item cppcheck
\begin{DoxyItemize}
\item With {\itshape make cppcheck} you can run static code analysis
\end{DoxyItemize}
\item Nginx
\begin{DoxyItemize}
\item Server which does post request for data and posts controll data
\end{DoxyItemize}
\item Python
\begin{DoxyItemize}
\item Webserver
\end{DoxyItemize}
\item Mongodb
\begin{DoxyItemize}
\item Databases
\end{DoxyItemize}
\item Open\+CV
\begin{DoxyItemize}
\item Capturing image from camera
\item Pattern recognition 
\end{DoxyItemize}
\end{DoxyItemize}