\href{https://travis-ci.org/danvk/dygraphs}{\tt } \href{https://coveralls.io/r/danvk/dygraphs}{\tt } \hypertarget{md_src_website_node_modules_dygraphs_README_autotoc_md19}{}\section{dygraphs Java\+Script charting library}\label{md_src_website_node_modules_dygraphs_README_autotoc_md19}
The dygraphs Java\+Script library produces interactive, zoomable charts of time series\+:



Learn more about it at \href{http://www.dygraphs.com}{\tt dygraphs.\+com}.

Get help with dygraphs by browsing the on \href{http://stackoverflow.com/questions/tagged/dygraphs?sort=votes&pageSize=50}{\tt Stack Overflow} (preferred) and \href{http://groups.google.com/group/dygraphs-users}{\tt Google Groups}.\hypertarget{md_src_website_node_modules_dygraphs_README_autotoc_md20}{}\subsection{Features}\label{md_src_website_node_modules_dygraphs_README_autotoc_md20}

\begin{DoxyItemize}
\item Plots time series without using an external server or Flash
\item Supports \href{http://dygraphs.com/tests/legend-values.html}{\tt error bands} around data series
\item Interactive \href{http://dygraphs.com/tests/link-interaction.html}{\tt pan and zoom}
\item Displays values \href{http://dygraphs.com/tests/legend-values.html}{\tt on mouseover}
\item Adjustable \href{http://dygraphs.com/tests/temperature-sf-ny.html}{\tt averaging period}
\item Extensive set of \href{http://www.dygraphs.com/options.html}{\tt options} for customization.
\item Compatible with the \href{http://dygraphs.com/data.html#datatable}{\tt Google Visualization A\+PI}
\end{DoxyItemize}

\#\# Minimal Example 
\begin{DoxyCode}
<html>
<head>
<script type="text/javascript" src="dygraph.js"></script>
<link rel="stylesheet" href="dygraph.css" />
</head>
<body>
<div id="graphdiv"></div>
<script type="text/javascript">
  g = new Dygraph(
        document.getElementById("graphdiv"),  // containing div
        "Date,Temperature\(\backslash\)n" +                // the data series
        "2008-05-07,75\(\backslash\)n" +
        "2008-05-08,70\(\backslash\)n" +
        "2008-05-09,80\(\backslash\)n",
        \{ \}                                   // the options
      );
</script>
</body>
</html>
\end{DoxyCode}


Learn more by reading \href{http://www.dygraphs.com/tutorial.html}{\tt the tutorial} and seeing demonstrations of what dygraphs can do in the \href{http://www.dygraphs.com/gallery}{\tt gallery}. You can get {\ttfamily dygraph.\+js} and {\ttfamily dygraph.\+css} from \href{https://cdnjs.com/libraries/dygraph}{\tt cdnjs} or \href{https://www.npmjs.com/package/dygraphs}{\tt from N\+PM} (see below).\hypertarget{md_src_website_node_modules_dygraphs_README_autotoc_md21}{}\subsection{Usage with a module loader}\label{md_src_website_node_modules_dygraphs_README_autotoc_md21}
Get dygraphs from N\+PM\+: \begin{DoxyVerb}npm install dygraphs
\end{DoxyVerb}


You\textquotesingle{}ll find pre-\/built JS \& C\+SS files in {\ttfamily node\+\_\+modules/dygraphs/dist}. If you\textquotesingle{}re using a module bundler like browserify or webpack, you can import dygraphs\+:


\begin{DoxyCode}
import Dygraph from 'dygraphs';
// or: const Dygraph = require('dygraphs');

const g = new Dygraph('graphdiv', data, \{ /* options */ \});
\end{DoxyCode}


Check out the \href{https://github.com/danvk/dygraphs-es6}{\tt dygraphs-\/es6 repo} for a fully-\/worked example.\hypertarget{md_src_website_node_modules_dygraphs_README_autotoc_md22}{}\subsection{Development}\label{md_src_website_node_modules_dygraphs_README_autotoc_md22}
To get going, clone the repo and run\+: \begin{DoxyVerb}npm install
npm run build
\end{DoxyVerb}


Then open {\ttfamily tests/demo.\+html} in your browser.

Read more about the dygraphs development process in the /\+D\+E\+V\+E\+L\+OP.md \char`\"{}developer guide\char`\"{}.\hypertarget{md_src_website_node_modules_dygraphs_README_autotoc_md23}{}\subsection{License(s)}\label{md_src_website_node_modules_dygraphs_README_autotoc_md23}
dygraphs is available under the M\+IT license, included in L\+I\+C\+E\+N\+S\+E.\+txt. 